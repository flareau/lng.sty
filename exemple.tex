\documentclass{article}
\usepackage{lng}
\usepackage{multicol}
\usepackage[moderate]{savetrees}

\author{François Lareau\\OLST, Université de Montréal}
\title{Macros \LaTeX\ pour linguistes}

\begin{document}
\maketitle

\section{Conventions d'écriture}

\begin{itemize}
	\item Agrammatical: \ungr\form{Il n'a pas le fait}
	\item Douteux: \doubt\form{Il s'est en allé}
	\item Sémantiquement bizarre: \infel\form{Tu manges une patate très douce}
	\item Forme linguistique: \form{Il pleuvait}
	\item Concept: \conc{pluie}
	\item Sens: \sem{pluie}
	\item Signe linguistique: \sign{pleuv--}\comb\sign{--ait}
	\item Lexie: \lex{pleuvoir}
	\item Grammème: \lex{pleuvoir}.\gramm{imp}.\gramm{3sg}
	\item Locution: \idiom{\lex{tomber des cordes}}, \form{Il \idiom{tombait des cordes}}
	\item Partie du discours: \lex{souper}\pos{n}
	\item Présupposé: \sem{\presup{X ayant autorité sur Y} X dit\ldots}
	\item Numéro de lexie: \lex{pleuvoir}\sensenum{I.1}
	\item Numéro de vocable: \lex{table}\vocnum{1}
	\item Nom de relation: \lex{il} est l'actant \rel{I} de \lex{pleuvoir}
	\item Relations étiquetées: \form{Il \lrel{suj} pleuvait \rrel{mod} souvent}
	\item Slash dans les exemples: \form{Il pleuvait\slashup neigeait\slashup grêlait}
\end{itemize}


\section{Exemples linguistiques}

\ex. \label{ex:silence}\form{Il règne un très lourd silence dans la pièce.} \hfill(source)\\
		 \smaller\lex{il}\pos{expl} \lex{régner}.\gramm{prés}.\gramm{3sg} \lex{un}.\gramm{masc}.\gramm{sg} \lex{très} \lex{lourd}.\gramm{masc}.\gramm{sg} \lex{silence}.\gramm{sg} \lex{dans} \lex{le}.\gramm{fem}.\gramm{sg} \lex{pièce}.\gramm{sg} \\
		 \sem{The room is dead silent}


\section{Définitions lexicographiques}

\lexdef{$x$ risque\sensenum{II.2a} de $y$-er}{$y$ est fort probable \presup{$y(x)$ étant non souhaitable pour l'énonciateur ou pour $x$}}


\newpage
\section{Fonctions lexicales}

\begin{multicols}{3}
\begin{itemize}
	\item \Syn, \Q\Syn
	\item \Anti, \Non
	\item \Conv{i}
	\item \Contr
	\item \Gener
	\item \Figur
	\item \So, \Vo, \Ao, \Advo
	\item \Claus
	\item \Pred
	\item \Si{i}
	\item \Equip, \CapLF
	\item \Sinstr, \Smed, \Smod, \Sloc, \Sres
	\item \Ai{i}
	\item \Able{i}, \Qual{i}
	\item \Advi{i}
	\item \Sing, \Mult
	\item \Imper
	\item \Perf, \Imperf
	\item \Result{i}
	\item \Germ, \Culm
	\item \Epit, \Redun
	\item \Magn, \Ver, \Bon
	\item \Plus, \Minus
	\item \Degrad
	\item \Locin, \Locad, \Locab
	\item \Instr
	\item \Propt, \Propti{i}
	\item \Copul
	\item \Oper{i}, \Func{i}, \Labor{ij}
	\item \Real{i}, \Fact{i}, \Labreal{ij}
	\item \Prepar, \Prox
	\item \Incep, \Fin, \Cont
	\item \Obstr, \Stop, \Excess
	\item \Caus, \Causi{i}
	\item \Liqu, \Liqui{i}
	\item \Perm, \Permi{i}
	\item \Son
	\item \Manif
	\item \Involv
	\item \Sympt{i}
	\item \lexfn{NonStd}
	\item \Q\Anti\Magn+\Anti\Ver
\end{itemize}
\end{multicols}




\section{Structures de traits}

\lexentry{départ}{
	\begin{avm}
		\[{}	pdd & N \\
					sens & \sem{partir} \\
					genre & masc \\
					tr &	\[{}	I & \[{} arg & 1 \\ pdd & N \\ prép & \lex{de}\] \] \]
	\end{avm}
}


\section{Alphabet phonétique international}

\begin{itemize}
	\item Voyelles: \textipa{[iIeEa yY\o\oe{} @ uUoOA \~e\~E\~a\~\o\~\oe{}\~O\~A:]}
	\item Consonnes: \textipa{[pt\t*{ts}\t*{tS}kbd\t*{dz}\t*{dZ}g fsSvzZ mn\ng{} lrKR]}
	\item Semi-consonnes: \textipa{[j4w]}
\end{itemize}

\section{Arbres syntaxiques}

\subsection{Dépendances}

\begin{dependency}[theme=simple,segmented edge]
	\begin{deptext}
		Ma \& nouvelle \& gardienne \& étudie \& le \& droit \& à \& l \& UdeM \\
	\end{deptext}
	\deproot{4}{RACINE}
	\depedge{3}{1}{dét}
	\depedge{3}{2}{mod}
	\depedge{4}{3}{suj}
	\depedge{4}{6}{dir}
	\depedge{6}{5}{dét}
	\depedge{4}{7}{obl}
	\depedge{7}{9}{cprép}
	\depedge{9}{8}{dét}
\end{dependency}


\subsection{Constituants}

\Tree [.Ph [.SN [.N Il ] ] [.SV [.V règne ] [.SN [.Dét un ] [.SA [.SAdv [.Adv très ] ] [.Adj lourd ] ] [.N silence ] ] [.SP [.P dans ] [.SN [.Dét la ] [.N pièce ] ] ] ] ]

\end{document}
